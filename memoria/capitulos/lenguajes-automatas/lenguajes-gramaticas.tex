% !TeX root = ../tfg.tex
% !TeX encoding = utf8

\chapter{Lenguajes y gramáticas}\label{chap:leng}

Antes de poder definir el lenguaje de la herramienta, es necesario introducir la teoría de lenguajes formales. Esta 
teoría tal y como se la conoce en la actualidad comenzó su desarrollo durante el siglo XX. Durante las décadas de 1930 
y 1940, los trabajos de Turing y Post desarrollaron ciertos conceptos de la teoría de lenguajes, como las cadenas de 
producción. Poco más tarde, Warren Weaver y A. D. Booth realizaron estudios sobre los lenguajes formales como 
herramientas para la traducción automática mediante computadores. No es hasta 1959 con el trabajo de Noam Chomsky en 
el que se describe la ``Jerarquía de Chomsky'' cuando la teoría de los lenguajes formales gana popularidad y se 
desarrolla \cite{greibach_1981_formal}.

\vspace{10pt}
A los largo de los siguientes capítulos hablaremos de los lenguajes regulares, interesantes desde el punto de vista 
práctico porque pueden ser usados para especificar la construcción de analizadores léxicos \cite{kelley_2001}. Estos 
analizadores léxicos nos permitirán transformar un archivo con código escrito en nuestro lenguaje en una secuencia de 
palabras o \textit{tokens} con los que construiremos las sentencias a ejecutar. Daremos a conocer también los autómatas
finitos, que serán las máquinas capaces de validar e interpretar estos lenguajes.

\vspace{10pt}
Después, daremos a conocer también los lenguajes y gramáticas libres de contexto, a partir de los cuales podremos 
definir los elementos del lenguaje de la herramienta, así como sus construcciones sintácticas. También definiremos a
los autómatas con pila, máquinas que, análogamente a los autómatas finitos, serán los capaces de interpretar este tipo
de lenguajes. Será con uno de estos autómatas con lo que construiremos el intérprete de nuestro lenguaje.

\vspace{10pt}
Vamos, en primer lugar, a definir algunos conceptos básicos de la teoría de lenguajes que utilizaremos a lo largo del 
capítulo y en capítulos posteriores.

\begin{definicion}[Alfabeto] Un alfabeto $\Sigma$ es un conjunto finito no vacío. Cada uno de los elementos de un 
alfabeto se denomina símbolo.
\end{definicion}
\begin{definicion}[Palabra] Una \textit{palabra} o \textit{cadena} es un conjunto finito de símbolos de un alfabeto. 
La palabra que contiene cero símbolos se denomina \textit{palabra vacía} y se denota por $\lambda$.
\end{definicion}

Sea $w$ una palabra, entonces $w^n$ es una palabra tal que \cite{Yu1997}:
\begin{enumerate}[i]
    \item $w^0=\lambda$.
    \item $w^n=ww^{n-1}$, para $n>0$.
\end{enumerate}
Además, tomando la palabra $w$ como $w=a_1a_2\dotsc a_n$, podemos definir la \textit{imagen inversa} de $w$, 
$w^{-1}$, de la siguiente manera: $w^{-1}=a_na_{n-1}\dotsc a_1$ \cite{Becerra2018}.

\begin{definicion}[Clausura transitiva o de Kleene] La \textit{clausura transitiva} o \textit{clausura de Kleene}
de un alfabeto $\Sigma$ ($\Sigma^*$) es el conjunto de todas las palabras de $\Sigma$. El conjunto de todas las palabras
de $\Sigma$ excluyendo la palabra vacía se denota como $\Sigma^+$.
\end{definicion}
\begin{definicion}[Concatenación] Se define la \textit{concatenación} de dos palabras como la aplicación
$\circ:\Sigma^*\times\Sigma^*\to\Sigma^*$ tal que $x\circ y=xy$, con $x, y\in\Sigma^*$.
\end{definicion}

Dado un alfabeto $\Sigma$, tomando la concatenación como operación binaria y $\lambda$ como elemento neutro,
$(\Sigma^*, \circ)$ es un monoide \cite{Mateescu1997}.

\begin{definicion}[Longitud] Se define la \textit{longitud} de una palabra $w$, denotada por $|w|$ como el número de
símbolos que componen la palabra.
\end{definicion}

% -------------------------------------------------------------------------------------------------------------------
\section{Lenguajes formales}

Con esas definiciones básicas, podemos ya definir el concepto de \textbf{\textit{lenguaje formal}}. Existen muchos 
tipos de lenguajes, diferenciados entre sí por su construcción y limitaciones.

\begin{definicion}[Lenguaje formal] Un \textit{lenguaje formal} o \textit{lenguaje} $L$ sobre un alfabeto $\Sigma$ es
un subconjunto de $\Sigma^*$. El \textit{lenguaje vacío} es el lenguaje formado por la palabra vacía, y el
\textit{lenguaje universal} sobre $\Sigma$ es $\Sigma^*$.
\end{definicion}

Sobre un lenguaje, se pueden definir las operaciones usuales de conjuntos: complemento, unión e intersección. Además,
el concepto de \textit{concatenación} puede extenderse a los lenguajes de la siguiente manera:
\begin{equation}
    L_1L_2=\{w_1w_2\mid\; w_1\in L_1, w_2\in L_2\}
\end{equation}

También se extiende la notación $L^n$, definida por \cite{Yu1997}:
\begin{enumerate}[(i)]
    \item $L^0=\{\lambda\}$.
    \item $L^n=L^{n-1}L$, para $n>0$.
\end{enumerate}

Un lenguaje $L$ es \textit{no numerable} si y sólo si existe un entero $n$ tal que, para todas palabras $x, y, z\in L$,
$xy^nz\in L\iff xy^{n+1}z\in L$ \cite{Mateescu1997}.

\begin{definicion}[Clausura de Kleene de un lenguaje] La \textit{clausura de Kleene} de un lenguaje $L$, denotada por
$L^*$, es la unión de todos exponentes no negativos de $L$, es decir:
\begin{equation}
    \bigcup_{i=0}^\infty L^i
\end{equation}
Asimismo, se define $L^+=\displaystyle\bigcup_{i=1}^\infty L^i$ 
\end{definicion}

La \textit{clausura de Kleene} es una operación que permite producir lenguajes infinitos a partir de finitos. Veamos
ahora una propiedad de esta operación y los lenguajes formales, que demuestra que para lenguajes sobre un alfabeto con
un único símbolo, su clausura es bastante simple.

\begin{teorema} Para cada lenguaje $L$ sobre un alfabeto $\{a\}$, existe un lenguaje finito $L_F\subseteq L$ tal que
$L^*=L_F^*$ \cite{Mateescu1997}.
\end{teorema}
\begin{proof}
Si $L$ es finito, entonces $L_F=L$. Sea entonces $L$ infinito y $a^p$, $p\geq 1$, la palabra más pequeña no vacía en 
$L$. Claramente, $\{a^p\}^*\subseteq L^*$, ya que como $a^p\in L$, entonces todos los exponentes de $a^p$ están en
$L^*$. Si se da la igualdad (que se da cuando $p=1$), es decir, $\{a^p\}^*=L^*$, entonces $L_F=\{a^p\}$. En caso
contrario, sea $a^{q_1}$ la palabra más pequeña de la diferencia $L^*-\{a^p\}^*$. Podemos escribir entonces $q_1$ como:
\begin{equation}
    q_1=t_1p+r_1,\;0<r_1<p,t_1\geq 1
\end{equation}
Podemos ver de nuevo que $\{a^p, a^{q_1}\}^*\subseteq L^*$. Si se da la igualdad, entonces $L_F=\{a^p, a^{q_1}\}$. De
no ser así, continuamos el proceso tomando $a^{q_2}$ como la palabra más pequeña de la diferencia 
$L^*-\{a^p, a^{q_1}\}^*$, siendo $q_2$ de la forma
\begin{equation}
    q_2=t_2p+r_2,\;0<r_2<p_2, r_2\neq r_1, t_2\geq 1
\end{equation}
Siguiendo este proceso, en el paso $k$-ésimo tomamos la palabra $a^{q_k}$, con
\begin{equation}
    q_k=t_kp+r_k,\;0<r_k<p, r_k\neq r_1, r_2, \dots, r_{k_1}, t_k\geq 1
\end{equation}
Puesto que solamente puede haber $p$ restos $r_i$, incluido el $0$ cuando se toma $p$, entonces tenemos que
\begin{equation}
    \{a^p, a^{q_1}, \dots, a^{q_s}\}^*=L^*,\;\text{para algún } s<p.
\end{equation}
\end{proof}

Veamos ahora una propiedad que nos da una idea sobre la magnitud de posibles lenguajes sobre un alfabeto.

\begin{teorema}El conjunto de todos los lenguajes sobre un alfabeto $\Sigma$ no es numerable.
\end{teorema}
\begin{proof}
Supongamos que el conjunto de todos los lenguajes sobre $\Sigma$ es numerable, y llamémoslo $\mathbb{L}$. Puesto que
$\mathbb{L}$ es numerable, puede ser enumerado de la forma $L_0,L_1,L_2,\dots$. Sabemos que $\Sigma^*$ es numerable,
puesto que los alfabetos son finitos. Por tanto, puede ser enumerado como $w_0,w_1,w_2,\dots$\newline
Sea $A=\{w_i\mid w_i\notin L_i\}$, es decir, $A$ es el conjunto de palabras que no pertenecen al lenguaje que tiene el 
mismo índice que ellas. Es claro que $A$ es un lenguaje sobre $\Sigma$, de aquí que $A=L_k$ para algún $k$. Observamos 
que si $w_k\in A$, entonces $w_k\notin L_k=A$. Por tanto, $w_k$ está y no está a la vez en $L_k$, llegando a una 
contradicción. Por otro lado, si $w_k\notin A$, entonces, por la definición de $A$ tenemos que $w_k\in L_k=A$,
llegando de nuevo a una contradicción.\newline
Puesto que $w_k$ debe ser o no un elemento de $A$, tenemos que la suposición es falsa, por lo que el conjunto es no
numerable \cite{kelley_2001}.
\end{proof}

Con esta propiedad, podemos ver que hay una cantidad innumerable de lenguajes que se pueden especificar sobre un
alfabeto. No existe pues ningún método que los defina a todos a la vez.

% -------------------------------------------------------------------------------------------------------------------
\section{Gramáticas formales}

Si bien los lenguajes se pueden especificar directamente, las gramáticas nos proporcionan otra forma de especificarlos
y de clasificarlos. Una gramática es un mecanismo finito mediante el cual es posible generar las palabras de un 
lenguaje \cite{Becerra2018}.

\begin{definicion}[Gramática formal] Una \textit{gramática formal} o \textit{gramática} $G$ es una tupla $(N,T,S,P)$
donde:
\begin{itemize}
    \item $N$ un alfabeto, denominado \textit{alfabeto no terminal}.
    \item $T$ otro alfabeto, denominado \textit{alfabeto terminal}.
    \item $S$, $S\in N$, es un símbolo denominado \textit{símbolo inicial} o \textit{axioma}.
    \item $P$ es el conjunto de \textit{reglas de producción}. Una regla de producción es una pareja $(w,v)$ tal que
    $w,v\in(N\cup T)^*$ y $w$ contiene al menos una letra de $N$. Se denotan normalmente como $w\to v$.
\end{itemize}
Además, $N\cap T=\emptyset$ \cite{Becerra2018}.
\end{definicion}

Veamos con un ejemplo esta definición. Tomando $N=\{A,S\}$ y $T=\{a,b\}$, definimos las reglas de producción:
\begin{align}
    S &\to ab & AS &\to bSb & A &\to\lambda \\
    S &\to aASb & A&\to bSb & aASAb &\to aa
\end{align}
Entonces, si $P$ es el conjunto con las reglas de producción definidas y $S$ el símbolo inicial, tenemos que 
$G=(N,T,S,P)$ es una gramática.

\vspace{10pt}
Para facilitar la escritura de las reglas de producción, podemos agrupar las reglas de producción que comparten
símbolo a la izquierda mediante líneas verticales. Por ejemplo, para el ejemplo anterior, podemos compactar las
reglas para $S$ como $S\to ab\mid aASb$.

\begin{definicion}[Vocabulario total]Para una gramática $G$, se dice que $\Sigma_G=N\cup T$ es el 
\textit{vocabulario total} de la gramática \cite{harrison_1978}. 
\end{definicion}

\begin{definicion}Sea $G=(N,T,S,P)$ una gramática y sean $\alpha,\beta\in\Sigma_G^*$. Se dice que $\alpha$
\textit{genera directamente} $\beta$, denotado por $\alpha\Rightarrow\beta$, si existen 
$\alpha_1,\alpha_2,\alpha',\beta'\in\Sigma_G^*$ tal que $\alpha=\alpha_1\alpha'\alpha_2$, 
$\beta=\alpha_1\beta'\alpha_2$ y $\alpha'\to\beta'$ es un elemento de $P$. Escribimos $\xRightarrow{*}$ para la
clausura reflexiva-transitiva de $\Rightarrow$ \cite{harrison_1978}.    
\end{definicion}

Volviendo al ejemplo anterior, podemos realizar la siguiente secuencia
\begin{gather}\label{eq:derivacion_ejemplo}
    S\Rightarrow a\underline{AS}b \underbrace{\Rightarrow}_{AS\Rightarrow bSb}ab\underline{S}bb
    \underbrace{\Rightarrow}_{S\Rightarrow ab} ababbb \\
    S\xRightarrow{*}(ab)^2b^2
\end{gather}

A la secuencia \eqref{eq:derivacion_ejemplo} se la denomina \textit{derivación} \cite{harrison_1978}.

\begin{definicion}Sea $G=(N,T,S,P)$ una gramática. El conjunto de \textit{sentencias} de $G$, denotado por $S(G)$ es
el conjunto
\begin{equation}
    S(G)=\{\alpha\in\Sigma_G^*\mid S\xRightarrow{*}\alpha\}
\end{equation}
es decir, el conjunto de palabras derivables desde $S$ \cite{harrison_1978}.
\end{definicion}

Tenemos, de esta manera, una forma de describir lenguajes a partir de una gramática y un alfabeto, mediante la
siguiente definición.

\begin{definicion}[Lenguaje generado]Sea $G=(N,T,S,P)$ una gramática. El \textit{lenguaje generado} por $G$, $L(G)$,
es el conjunto
\begin{equation}
    L(G)=S(G)\cap T^*=\{w\in T^*\mid S\xRightarrow{*}w\}
\end{equation}
esto es, el conjunto de de derivaciones a partir de $S$ que terminan en palabras del alfabeto terminal 
\cite{harrison_1978}.
\end{definicion}

\endinput
%--------------------------------------------------------------------
% FIN DEL CAPÍTULO. 
%--------------------------------------------------------------------
