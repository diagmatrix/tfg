% !TeX root = ../tfg.tex
% !TeX encoding = utf8

\chapter{Conclusiones}

Al terminar el trabajo, quedan algunos objetivos por cumplir, principalmente la finalización del intérprete. Sin embargo,
se ha conseguido realizar, en primer lugar, una librería en Scala que permite automatizar el proceso de aprendizaje de
redes neuronales mediante algoritmos distribuidos. También se ha conseguido generar una gramática para el lenguaje de la
herramienta, así como el analizador léxico y el \textit{parser} del intérprete.

\vspace{10pt}
Es interesante resaltar la diferencia en la sintaxis de ese lenguaje frente al usual para las técnicas de ML, Python.
Las restricciones que presenta Scala con su tipado fuerte, frente al de Python, junto con el énfasis de Scala en la
inmutabilidad y el paradigma de programación funcional, parecen resultar, más que un impedimento, una ayuda involuntaria
para mejorar la fiabilidad de un modelo de aprendizaje automático. Para las redes neuronales, que presentan una cierta
complejidad a la hora de su configuración y requieren una inversión en tiempo para su entrenamiento, es importante
asegurarse de la validez del código escrito antes de su funcionamiento. Es por ello que un lenguaje como Scala, que
requiere de un trabajo previo a la ejecución más concienzudo puede ser un buen candidato como lenguaje a utilizar para
el aprendizaje automático.

\vspace{10pt}
La aproximación a los lenguajes de programación y los intérpretes desde un punto de vista formal ha dado una visión distinta
del funcionamiento y componentes de los mismos. El visualizar un lenguaje de programación como el generado por una gramática
libre de contexto permite entenderlos desde una perspectiva más abstracta que permite entender los patrones generales de
diseño de los mismos.

\vspace{10pt}
Finalmente, cabe destacar que el objetivo último de este trabajo es intentar democratizar de alguna manera las técnicas de
ML para su uso por personas sin conocimientos específicos. Dada la tendencia actual de adopción en masa por todos los
sectores de la sociedad, una herramienta así puede ser de gran ayuda para limitar el impacto que pueden provocar en las
personas más alejadas de este campo de la ingeniería.

% -------------------------------------------------------------------------------------------------------------------
\section{Futuros trabajos}

Aunque se han alcanzado algunos de los objetivos propuestos, no se ha podido completar la herramienta originalmente
ideada para el problema a resolver. Por tanto, el principal trabajo futuro a realizar es la finalización del intérprete
de SSDL.

\vspace{10pt}
Sería deseable también ampliar el número de algoritmos distribuidos de la librería implementada. Puesto que se ha 
construido una interfaz para ellos, no debería ser muy complicado implementar nuevos algoritmos (sin tener en cuenta la
dificultad de implementación inherente que puedan tener). Para incluirlos en el lenguaje, simplemente tendría que 
expandirse la producción \textlangle object\textrangle, añadiendo en nombre del algoritmo como posible parte derecha y
añadir producciones para los parámetros específicos del mismo.

\vspace{10pt}
De la misma manera, un futuro trabajo podría ser la ampliación de tipos de modelos de aprendizaje automático soportados
por la herramienta. Actualmente solamente es capaz de expresar redes neuronales prealimentadas con una capa oculta, pero
se podría generalizar a redes neuronales prealimentadas, y de ahí a otro tipo de redes más complejas, como las 
convolucionales. El procedimiento para incorporarlas al lenguaje sería similar: habría que expandir de nuevo la producción
de \textlangle object\textrangle, así como crear las producciones con los parámetros específicos de ese tipo de redes.

\endinput
%--------------------------------------------------------------------
% FIN DEL CAPÍTULO. 
%--------------------------------------------------------------------
